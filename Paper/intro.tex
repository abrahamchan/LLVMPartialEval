\section{Introduction}

Software systems are steadily increasing in complexity, driven in large part by the ongoing demand for technological development and growing business requirement for various software, which implicitly increases software development complexity to meet all demands~\cite{Lyu}. 
As a result of this, software optimization has become increasingly popular and important, which aims to make certain aspects of software work more efficiently with fewer resources to meet growing business demands.
At the application level, compiler optimizations have been widely deployed to streamline program performance and conserve memory space~\cite{second}.
Particularly, the technique of partial evaluation has been explored as an aggressive form of program optimization by specialization~\cite{Jones}.
Partial evaluations reduce the program space to its minimal form required to yield a correct output from specific program inputs.
While partial evaluations guarantee program correctness, it is not the only desired attribute in critical software systems.
Fault tolerance is also an important component in many mission-critical systems that are expected to operate under extreme conditions~\cite{Somani}.
To the best of our knowledge, no existing studies explore the fault tolerance capability of programs optimized as partial evaluations.
We aim to implement a LLVM compiler optimization pass to support partial evaluation and empirically study the impact on error resiliency in the optimized programs through fault injection experiments.
