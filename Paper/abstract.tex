\begin{abstract}
Fault tolerant program optimizations have become increasingly important as software systems undertake higher complexity applications in various fields including mission-critical systems.
Partial evaluation is an aggressive form of program optimization that exploits known inputs to specialize a program in the hopes of delivering performance gain.
In this paper, we empirically evaluate the effects of partial evaluations on program error resilience.
We first present a simplified implementation of partial evaluation.
We then evaluate its impact on error resilience through a campaign of software implemented fault injection experiments using different fault types across three benchmark programs.
Finally, we perform a statistical comparison between the error resiliencies of the program and its partial evaluation.
Experiment results show that partial evaluations do not typically impact a program's fault tolerance.
\end{abstract}