\section{Methodology}
\label{sec:methodology}

\subsection{Motivating Example}
Consider the following code snippet.
The figure on the left shows the unoptimized program, while the figure on the right shows the partial evaluation optimization.
Notice that the function call to foo2 is replaced with a constant instead.
Note that the actual transformation is performed at the IR level, not through source code.

\subsection{Approach}
A number of compiler-level partial evaluation strategies exist.
We will implement one form of partial evaluation as a LLVM transformation pass.
Our pass will apply static analysis to convert function calls to constants, by evaluating branch and loop conditions with function inputs at compile time.
This simplified approach was presented by Fujita’s work on partial evaluations~\cite{Fujita}.
Programs in C/C++ will be compiled to LLVM IR using the Clang front-end and the partial evaluation optimization pass will be applied subsequently on the generated IR.
Then, we will compile the optimized program to an executable, in order to dynamically confirm program correctness.
After program correctness is verified, we will measure the number of IR instructions reduced as a result of the partial evaluation optimization.
Finally, we will evaluate and compare the error resilience of the program’s partial evaluation with its original unoptimized implementation through a series of fault injection experiments.
