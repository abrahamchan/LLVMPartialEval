\section{Evaluation}

We aim to evaluate the error resilience of program partial evaluations through a series of fault injection experiments using LLFI~\cite{LLFI}, a LLVM based tool.
LLFI injects a variety of hardware and software faults at the LLVM IR level using software implemented fault injection.
For this project, we will focus on bit flips, stuck-ats, buffer overflows, function argument corruptions and invalid pointer corruptions.
These faults broadly represent a range of common hardware and software faults~\cite{V2005}, including permanent and transient faults. 

We plan to run 1000 fault injections of each fault type on several C/C++ applications, drawn from the Parboil~\cite{Parboil}, and SPEC CPU2006 benchmarks~\cite{SPEC} respectively.
We will group the effects of injected faults on a program in the following failure modes: Silent Data Corruptions (SDCs) and Crashes/Hangs.
We will then report the failure rates (i.e SDCs / 1000) for each failure mode by fault type.
This process is performed once for the unoptimized program (base case) and again for the program’s partial evaluation.
We hypothesize that the mean failure rate of a program’s partial evaluation is statistically equal to the mean failure rate of the same unoptimized program.
Using the mean failure rates from the two samples, we will perform a t-test to determine whether the means are statistically equivalent to each other, within a 95\% confidence interval.
We consider the error resilience of program partial evaluations to be acceptable if the hypothesis holds across failure modes and fault types. 

80\% Goal: We implement the partial evaluation optimization pass with minor limitations on LLVM and perform a small number of fault injection experiments.

100\% Goal: We implement the partial evaluation optimization pass on LLVM and perform fault injection experiments on several benchmarks. We compare the error resiliency between the unoptimized program and its partial evaluation through hypothesis testing. 
