\section{Conclusion}
\label{sec:conclusion}

As programs increase in complexity, program optimizations are progressively deployed to improve program performance.
Partial evaluation is an aggressive form of program optimization that can be applied when the program inputs are known at compile time.
Despite the performance gains offered by partial evaluations, it remains unclear whether the partial evaluations degrade the error resilience of the original program.
We answer this question by implementing partial evaluation as a compiler pass and designing a fault injection experiment to statistically compare the error resilience of the partial evaluation with its original implementation.
Our results show that there is no statistical difference between their error resiliency over most fault types.
However, the validity of this observation depends on the injected fault type on a program.
In some cases, a partial evaluation may not be a satisfactory optimization for tolerating certain types of faults.
Therefore, we conclude that partial evaluations generally do not diminish a program's error resilience over most fault types. 
